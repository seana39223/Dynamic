\chapter{Code Examples}
\section{Registration AngularJS Controller (register.js)}
\begin{verbatim}
angular.module('register.controllers', [])
.controller('RegisterCtrl', function($scope, $ionicModal, 
$http, $state, $ionicPopup) {
  var apiReturns;
  var api = "http://seananderson.co.uk/api/listgenre.php";
  $http.get(api).then(function(res) {
    var length = (res['data']).length;
    var select = angular.element(document.querySelector('#genre'));
    for (i=0;i<length;i++) {
      select.append('<option>' + res['data'][i] + '</option>');
    }
  })

   // Form data for the login modal
  $scope.register = {};

  //Generic popup as there may need to be 
  //multiple popups for this page. Code is based of 
  //ionic documentation for popup.
  popUp = function(title, message) {
    var alertPopup = $ionicPopup.alert({
      title: title,
      template: message
    });
    alertPopup.then(function(res) {
    });
  }

  //Return to login page i.e user has decided they dont need to register.
  $scope.returnToLogin = function() {
    $state.go('login');
  }

  //Function which checks to make sure all fields 
  //on the form have been filled in.
  $scope.checkFields = function() {
    if ($scope.register.fName==undefined){
      popUp("Field Missing", "First Name Field Was Not Entered");
      return false;
    }

    if ($scope.register.lName==undefined) {
      popUp("Field Missing", "Last Name Field Was Not Entered.");
      return false;
    }

    if ($scope.register.email==undefined) {
      popUp("Field Missing", "Email address was not entered.");
      return false;
    }

    if ($scope.register.password==undefined) {
      popUp("Field Missing", "Password was not entered");
      return false
    }

    if ($scope.register.cPassword==undefined) {
      popUp("Field Missing", "Confirmation password was not entered");
      return false;
    }

    if ($scope.register.pCode==undefined) {
      popUp("Field Missing", "Postcode was not entered");
      return false;
    }

    if ($scope.register.dName==undefined) {
      popUp("Field Missing", "Display Name was not entered");
      return false;
    }

    if ($scope.register.userType==undefined) {
      popUp("Field Missing", "User Type was not selected.");
      return false;
    }

    if ($scope.register.genre===undefined) {
      popUp("Field Missing", "You haven't selected your 
      favoruite genre of music.");
      return false;
    }
    else {
      return true;
    }
  }

  //Function for regsitering genre of music which user likes.
  $scope.registerGenre = function() {
    var api = "http://seananderson.co.uk/api/registergenre.php";
    $scope.register.genre.forEach(function(genre) {
    var data = {
      email: $scope.register.email,
      genre: genre
      }
      $http.post(api, data).then(function(res) {
        console.log(res);
      })
    })
  }

  //Function which valides that the users postcode 
  //they have entered is correct.
  $scope.checkPostcode = function() {
    var postcode = $scope.register.pCode;
    $scope.register.pCode = postcode.replace(/[\s]/g, '');
    if ($scope.register.pCode.length!=6) {
      if ($scope.register.pCode.length!=7) {
        if ($scope.register.pCode.length!=8) {
          popUp("Postcode is incorrect length must be 
          6,7 or 8 characters.")
          return false;
        }
      }
    }
    return true;
  }

  $scope.checkPassword = function() {
    if ($scope.register.password.length<6) {
      popUp("Password Invalid", "Password length must be 6 or greater.");
      return false;
    }
    return true;
  }


  //Function for when the user clicks the validation button.
  $scope.doRegistration = function() {
    //Checks all necessary fields have values.
    if ($scope.checkFields()==false){
      return;
    }
    if($scope.checkPostcode()==false) {
      return;
    }

    if ($scope.checkPassword()==false) {
      return;
    }

    //Validates email address is a new email.
    var api= "http://seananderson.co.uk/api/checkemail.php";
    var data = {
      email: $scope.register.email
    }
    $http.post(api,data).then(function(res) {
      var apiResponse = JSON.stringify(res);
      var correct = "This email is fine"
      if (apiResponse.includes(correct)) {
        if ($scope.register.password==$scope.register.cPassword) {
          if($scope.register.userType=="Music Lover") {
            var type = 1;
          }
          else if ($scope.register.userType=="Artist") {
            var type = 2;
          }
          else if ($scope.register.userType=="Venue Owner") {
            var type = 3;
          }
          else {
            popUp('No User Type', 'No user type has been selected.' );
          }
          var api = "http://seananderson.co.uk/api/displayname.php";
          var data = {
            dName: $scope.register.dName
          }

          //Checks Display Name isn't already being used.
          $http.post(api, data).then(function(res) {
            apiReturns = JSON.stringify(res);
            var incorrect = "This user name has already been used";
            if (apiReturns.includes(incorrect)) {
              popUp('This display name has already been 
              used please choose another 
              display name.');
            }
            else {
              var api = "http://seananderson.co.uk/api/register.php";
              var data = {
                firstName: $scope.register.fName,
                lastName: $scope.register.lName,
                email: $scope.register.email,
                type: type,
                password: $scope.register.password,
                postcode: $scope.register.pCode,
                dName: $scope.register.dName
              }
              $http.post(api, data).then(function(res){
                apiReturns = JSON.stringify(res);
                if (apiReturns.includes(
                'New user added succesfully')>=0) {
                  localStorage.setItem('email', $scope.register.email);
                  localStorage.setItem('dName', $scope.register.dName);
                  if ($scope.registerGenre!=undefined) {
                    $scope.registerGenre();
                  }
                  $state.go('app.profile');
                  popUp('Welcome', 'Welcome to Dynamic please 
                  fill in your profile page and then 
                  follow some users');
                }
              })
            }
          })
        }
        else {
          popUp('Passwords do not match', 'The confirmation 
          password is not the same as the initial 
          password you entered');
        }
      }
      else {
        popUp("Email is not valid", "Please use a valid 
        email address which hasn't 
        already been used");
      }
    })
  }
})
\end{verbatim}

\section{Registration API (register.php)}
\begin{verbatim}
<?php
header('Access-Control-Allow-Origin: *');
header("Access-Control-Allow-Headers: Origin, 
X-Requested-With, Content-Type, Accept");
header('Access-Control-Allow-Methods: GET, POST, PUT');

include "config.php";
$connection = new mysqli($server, $dbuser, $dbpass, $dbname);

if ($connection->connect_error) {
    die ("Connection to the database failed." 
    . $connection->connect_error);
}

if ($_SERVER['REQUEST_METHOD'] == 'POST' && empty($_POST))
    $_POST = json_decode(file_get_contents('php://input'), true);

$firstName = mysqli_real_escape_string($connection, $_POST['firstName']);
$lastName = mysqli_real_escape_string($connection, $_POST['lastName']);
$email = mysqli_real_escape_String($connection, $_POST['email']);
$type = mysqli_real_escape_String($connection, $_POST['type']);
$dName = mysqli_real_escape_String($connection, $_POST['dName']);
$password = mysqli_real_escape_String($connection, $_POST['password']);
$password = password_hash($password, PASSWORD_DEFAULT);
$postcode = mysqli_real_escape_String($connection, $postcode);

if (filter_var($email, FILTER_VALIDATE_EMAIL)) {
    $sql = "INSERT into users (firstname, lastname, email, type, 
    displayname, password, postcode)
    VALUES ('$firstName', '$lastName', '$email', 
    $type, '$dName', '$password', '$postcode')";

    if ($connection->query($sql) === TRUE) {
        echo "New user added succesfully";
    } 
    else {
        echo "Error: " . $sql . "</br>" . $conn->error;
    }
}

else {
    echo "Email is not valid";
}

$connection->close();
?>
\end{verbatim}

\section{recommendedevents.php}
\begin{verbatim}
<?php
header('Access-Control-Allow-Origin: *');
header("Access-Control-Allow-Headers: Origin, X-Requested-With, 
Content-Type, Accept");
header('Access-Control-Allow-Methods: GET, POST, PUT');

include "../config.php";
$connection = new mysqli($server, $dbuser, $dbpass, $dbname);
if ($connection->connect_error) {
    die ("Connection to the database failed." . 
    $connection->connect_error);
}

if ($_SERVER['REQUEST_METHOD'] == 'POST' && empty($_POST)) {
    $_POST = json_decode(file_get_contents('php://input'), true);
}

$eventsArray = array();
$email = mysqli_real_escape_string($connection, $_POST['email']);
$sql = "SELECT id from users WHERE email = '$email'";
$userId = $connection->query($sql) or 
trigger_error($mysqli->error."[$sql]");
$array = $userId->fetch_array(MYSQLI_ASSOC);
$userId = $array['id'];
$sql = "SELECT * from event_fav WHERE user_id = '$userId'";
$eventsSql = $connection->query($sql) or 
trigger_error($mysqli->error."[$sql]");
$eventsAttending = array();
while($events = mysqli_fetch_assoc($eventsSql)) {
    $eventsAttending[] = $events;
}

//If the user does not have any events which they are 
//attending it uses the genre to establish which 
//events the user should attend.
if(empty($eventsAttending)) {
    $sql = "SELECT genre_id from user_genres 
    WHERE user_id = '$userId'";
    $genres = $connection->query($sql) or 
    trigger_error($mysqli->error."[$sql]");
    foreach ($genres as $genre) {
      $genreIdSql = $genre['genre_id'];
      $sql = "SELECT * from events WHERE genre_id = '$genreIdSql'";
      $events = $connection->query($sql) or 
      trigger_error($mysqli->error."[$sql]");
      while($event = mysqli_fetch_assoc($events)) {
          if( in_array( $event ,$eventsArray ) ) {
          }
          else {
              $eventsArray[] = $event;
          }
       }
    }
  print json_encode($eventsArray);
}

//If they do have events they are attending return 
//this so a different API can be called.
else{
  print json_encode("Call other machine learning API");
}
?>
\end{verbatim}

\section{Login Unit Test (logintest.js)}
\begin{verbatim}
describe('LoginCtrl', function() {

    var controller, 
        loginMock,
        stateMock,
        ionicPopupMock;
    
    beforeEach(module('app'));

    beforeEach(module('#'));  

    describe('#doLogin', function() {


        it('Should Login using correct details', function() {
            expect(loginMock.doLogin).
            toHaveBeenCalledWith('sea6@aber.ac.uk', 'b261la'); 
        });

        describe('when the login action is performed,', function() {
            it('if successful, should change state to home', function() {
                expect(stateMock.go).toHaveBeenCalledWith('app.home');
            });

            it('if unsuccessful, should show a popup', function() {

                expect(ionicPopupMock.alert).toHaveBeenCalled();
            });
        });
    })

    describe('#register', function(){

        it('Should go to register screen', function(){
             expect(stateMock.go).toHaveBeenCalledWith('register');
        })
    })
});
\end{verbatim}

